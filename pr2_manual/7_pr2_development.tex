\chapter{Writing Code on the PR2}

The PR2 comes with the latest released distribution
of ROS installed from binary debian packages in \texttt{/opt/ros}. You'll need to set up your personal ROS environment to develop on the PR2. 


\section{Installing code in user-space}

PR2 developers typically use the installed versions of ROS to operate the PR2 systems and perform low-level functions. Developers test and integrate their own code by setting up an ``overlay'' of their ROS packages and the default installation.

\subsection{rosinstall}

Use \texttt{rosinstall} to install ROS software in your home directory on the PR2. You can use \texttt{rosinstall} to set up your basic ROS environment.

There is more documentation on the wiki package page
\href{http://www.ros.org/wiki/rosinstall}{rosinstall}.

\subsubsection{Make rosinstall file}

You'll need to create a \texttt{rosinstall} file to store your own ROS packages, with installation instructions. See \href{http://www.ros.org/wiki/rosinstall#File\_format}{rosinstall file format} for details.

\begin{verbatim}
- VERSION-CONTROL-TYPE:
    uri: URI
    local-name: WORKING-COPY-NAME
\end{verbatim}
Recreate this as ~/custom.rosinstall substituting appropriately for the all caps sections. 

Then type:
\begin{verbatim}
~/rosinstall  ~/local_dir /opt/ros/cturtle ~/custom.rosinstall
\end{verbatim}

Now, add the following line to your \texttt{~/.bashrc} file to source your local ROS configuration.
\begin{verbatim}
. ~/local_dir/setup.sh
\end{verbatim}

\subsection{Creating a package}
This is documented on the wiki in the
\href{http://www.ros.org/wiki/roscreate}{roscreate package} tool.

\subsection{Using a local version of a stack}
If a different version of a stack is desired instead of an installed
one, add it to your overlay rosinstall file.  Stacks in overlays are
prepended to the path and thus will take priority over an existing
stack.  

{\bf Note:} If you modify any low-level stacks, you may need to rebuild any stacks that depend on them from source. For example, changes to \texttt{image\_common} may require that \texttt{stereo\_image\_proc} be rebuilt.

\section{Learning ROS}
\subsection{Documentation and Tutorials}
Documentation for ROS and the PR2 can be found on \href{http://www.ros.org}{ros.org}. An overview 
of the relevant libraries for filtering, navigation, cooridinate tranforms, etc., can be foung at 
\href{http://www.ros.org/APIs}{http://www.ros.org/APIs}. Below is a listing of several tutorials 
for learning and using ROS:
\begin{description}
\item[\href{http://www.ros.org/wiki/ROS/Tutorials}{ROS stack tutorials}] These tutorials cover the
basic ROS concepts and tools for writing and using ROS nodes.
\item[\href{http://www.ros.org/wiki/tf/Tutorials}{tf tutorials}] These tutorials cover using the tf 
library for cooridinate transofrms using the python and C++ APIs.
\item[\href{http://www.ros.org/wiki/navigation/Tutorials}{navigation tutorials}] These tutorials 
cover configuring and using the navigation stack on the PR2 and other robots.
\item[\href{http://www.ros.org/wiki/laser\_pipeline/Tutorials}{laser\_pipeline tutorials}] These tutorials 
cover processing laser data and converting laser data into 3D representations.
\item[\href{http://www.ros.org/wiki/image\_common/Tutorials}{image\_common tutorials}] These tutorials 
cover working with images in ROS.
\item[\href{http://www.ros.org/wiki/actionlib\_tutorials/Tutorials}{actionlib tutorials}] These 
tutorials cover implementing actions, a standard interface preemptible highlevel tasks, using 
the python and C++ APIs. More example tutorials can be found at 
\href{http://www.ros.org/wiki/turtle\_actionlib}{turtle actionlib}.
\end{description}

\section{ROS Help and Support}

\subsection{Mailing Lists}

For software or ROS questions, it is recommended to use mailings lists to contact developers.

\begin{description}
\item[\href{https://code.ros.org/mailman/listinfo/ros-users}{ros-users}] This mailing list 
provides the latest ROS news and is a forum for posting questions to the ROS community for help. 

\item[\href{http://lists.willowgarage.com/cgi-bin/mailman/listinfo/pr2-users}{pr2-users}] This 
mailing list provides the latest PR2 news and is a forum for posting questions to the PR2 community 
for help.

\item[\href{http://lists.willowgarage.com/cgi-bin/mailman/listinfo/pr2-admins}{pr2-admins}] This 
mailing list is for network/system administrators of the PR2. 

\end{description}

Please follow these general guidelines when posting mailing list messages:
\begin{description}
\item[Don't Ask General Programming Questions] The mailing lists are for ROS questions only.
\item[Be as clear as possible] Give very specific error messages. Give versions of relevant stacks, your operating system, and any other important details.
\item[Copy-Paste Error Messages] Copy and paste entire error messages into your mailing list entry or ticket.
\end{description}

More information is available at \href{http://www.ros.org/wiki/Support}{ROS Support}.

\subsection{Software Bugs/Tickets}

To report software problems, file a ticket against the software component.

\begin{description}
\item[\href{https://code.ros.org/trac/ros/newticket}{ROS Trac}] Report problems with the core ROS stacks only.

\item[\href{https://code.ros.org/trac/ros-pkg/newticket}{ROS-PKG Trac}] Report problems with any stack in ros-pkg (ex: camera\_drivers, laser\_drivers, common\_msgs, diagnostics).

\item[\href{https://code.ros.org/trac/wg-ros-pkg/newticket}{WG-ROS-PKG Trac}] Report problems with any stack in wg-ros-pkg (ex: pr2\_mechanism, pr2\_robot).
\end{description}
