\chapter{Writing code on PR2}
The PR2 comes with the PR2 varient of the latest released distribution
or ROS installed from binary debian packages in \texttt{/opt/ros} .  It is
recommended to use this installation for it saves compile time and
disk space.

\paragraph{Start by getting rosinstall}
There is more documentation on the wiki package page
\href{http://www.ros.org/wiki/rosinstall}{rosinstall}.

\begin{verbatim}
wget --no-check-certificate http://ros.org/rosinstall -O ~/rosinstall
chmod 755 ~/rosinstall
\end{verbatim}

\section{Installing code in user-space}
To setup your environment with the installed packages type:
\begin{verbatim}
. /opt/ros/boxturtle/setup.sh
\end{verbatim}
\subsection{Installing another repository in user-space}
To add another repository an overlay rosinstall file should be created
like the following:
\begin{verbatim}
- VERSION-CONTROL-TYPE:
    uri: URI
    local-name: WORKING-COPY-NAME
\end{verbatim}
Recreate this as ~/custom.rosinstall substituting appropriately for the all caps secionds. 

Then type:
\begin{verbatim}
~/rosinstall -o ~/local_dir ~/custom.rosinstall
\end{verbatim}
\subsubsection{Setting environment to use local\_dir}
\begin{verbatim}
. ~/local_dir/setup.sh
\end{verbatim}

\subsection{Creating a package}
This is documented on the wiki in the
\href{http://www.ros.org/wiki/roscreate}{roscreate package}

\subsection{Using a local version of a stack}
If a different version of a stack is desired instead of an installed
one, add it to your overlay rosinstall file.  Stacks in overlays are
prepended to the path and thus will take priority over an existing
stack.  Note be careful, if changing a lower level stack, for all
things which depend on it's changes must be rebuilt. This can be an
issue if the dependent stacks are also installed.

\section{Where to Start}
\subsection{Documetation and Tutorials}
Documentation for ROS and the PR2 can be found on \href{http://www.ros.org}{ros.org}. An overview 
of the relevant libraries for filtering, navigation, cooridinate tranforms, etc., can be foung at 
\href{http://www.ros.org/APIs}{http://www.ros.org/APIs}. Below is a listing of several tutorials 
for learning and using ROS:
\begin{description}
\item[\href{http://www.ros.org/wiki/ROS/Tutorials}{ROS stack tutorials}] These tutorials cover the
basic ROS concepts and tools for writing and using ROS nodes.
\item[\href{http://www.ros.org/wiki/tf/Tutorials}{tf tutorials}] These tutorials cover using the tf 
library for cooridinate transofrms using the python and C++ APIs.
\item[\href{http://www.ros.org/wiki/navigation/Tutorials}{navigation tutorials}] These tutorials 
cover configuring and using the navigation stack on the PR2 and other robots.
\item[\href{http://www.ros.org/wiki/laser\_pipeline/Tutorials}{laser\_pipeline tutorials}] These tutorials 
cover processing laser data and converting laser data into 3D representations.
\item[\href{http://www.ros.org/wiki/image\_common/Tutorials}{image\_common tutorials}] These tutorials 
cover working with images in ROS.
\item[\href{http://www.ros.org/wiki/actionlib\_tutorials/Tutorials}{actionlib tutorials}] These 
tutorials cover implementing actions, a standard interface preemptible highlevel tasks, using 
the python and C++ APIs. More example tutorials can be found at 
\href{http://www.ros.org/wiki/turtle\_actionlib}{turtle actionlib}.
\end{description}

\subsection{Mailing Lists}
\begin{description}
\item[\href{https://lists.sourceforge.net/lists/listinfo/ros-users}{ros-users}] This mailing list 
provides the latest ROS news and is a forum for posting questions to the ROS community for help. 
\item[\href{http://lists.willowgarage.com/cgi-bin/mailman/listinfo/pr2-users}{PR2-users}] This 
mailing list provides the latest PR2 news and is a forum for posting questions to the PR2 community 
for help.
\end{description}


