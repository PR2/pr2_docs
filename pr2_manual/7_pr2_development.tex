\chapter{Writing code on PR2}
\section{Installing code in user-space}
\subsection{Creating a package}
\subsection{Using a local version of a stack}
\section{Where to Start}
\subsection{Documetation and Tutorials}
Documentation for ROS and the PR2 can be found on \href{http://www.ros.org}{ros.org}. An overview 
of the relevant libraries for filtering, navigation, cooridinate tranforms, etc., can be foung at 
\href{http://www.ros.org/APIs}{http://www.ros.org/APIs}. Below is a listing of several tutorials 
for learning and using ROS:
\begin{description}
\item[\href{http://www.ros.org/wiki/ROS/Tutorials}{ROS stack tutorial}] These tutorials cover the
basic ROS concepts and tools for writing and using ROS nodes.
\item[\href{http://www.ros.org/wiki/tf/Tutorials}{tf tutorials}] These tutorials cover using the tf 
library for cooridinate transofrms using the python and C++ APIs.
\item[\href{http://www.ros.org/wiki/navigation/Tutorials}{navigation tutorials}] These tutorials 
cover configuring and using the navigation stack on the PR2 and other robots.
\item[\href{http://www.ros.org/wiki/laser_pipeline/Tutorials}{laser\_pipeline tutorials}] These tutorials 
cover processing laser data and converting laser data into 3D representations.
\item[\href{http://www.ros.org/wiki/image_common/Tutorials}{image\_common tutorials}] These tutorials 
cover working with images in ROS.
\item[\href{http://www.ros.org/wiki/actionlib_tutorials/Tutorials}{actionlib tutorials}] These 
tutorials cover implementing actions, a standard interface preemptible highlevel tasks, using 
the python and C++ APIs. More example tutorials can be found at 
\href{http://www.ros.org/wiki/turtle_actionlib}{turtle actionlib}.
\end{description}

\subsection{Mailing Lists}
\begin{description}
\item[\href{https://lists.sourceforge.net/lists/listinfo/ros-users}{ros-users}] This mailing list 
provides the latest ROS news and is a forum for posting questions to the ROS community for help. 
\item[\href{http://lists.willowgarage.com/cgi-bin/mailman/listinfo/pr2-users}{PR2-users}] This 
mailing list provides the latest PR2 news and is a forum for posting questions to the PR2 community 
for help.
\end{description}

Point users to documentation for all packages.

Point users to ros-users and pr2-users mailing lists

Point users to wiki.

Recommend a good set of tutorials.
