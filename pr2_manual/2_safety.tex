
\chapter{Safety}

Safety is a key goal at Willow Garage.  Safety is important yet challenging, and it is a continual process shared by the robot designer, user, and administrator.  In the following chapter, we provide an overview of the issues, describe some safety-related design features, enumerate a set of basic usage guidelines to support safety, and detail the safety program.

\section{Overview}

Safety is a vitally important concern whenever you are around a PR2. It’s a heavy piece of equipment with many moving parts. The robot travels through the environment and can carry and manipulate a wide variety of objects. Its movements and actions are not completely predictable. The PR2 can cause significant damage if it falls on or runs over someone. There are several ways it can pinch, grab, and twist fingers or other body parts. The robot can wield dangerous implements and knock heavy objects over. You must always be cautious and attentive when you are around a PR2.

Managing safety is a challenge when dealing with any complex engineering system.  In the case of the PR2, consider also the open, extensible, programmable, experimental nature of the platform.  The PR2's capabilities and behaviors change over time, with user interactions, and with re-programming.

With this in mind, Willow Garage has chosen a three-fold approach.  First, we have designed the PR2 to minimize potential risks and maximize inherent safety, cognizant of its uncertain uses.  Second, we communicate to all users about how to minimize risk. And third, we have implemented an explicit safety program to ensure that the community continues to identify potential hazards, seek design mitigations, and communicate effective usage guidelines.

\section{Design Features}

We have designed both hardware and software to minimize risks, while retaining the power of an open platform.  These aspects of the design are often described as inherent safety features.  For example, the PR2's arms are back-drivable.  This means that when an arm encounters an object, be it a table or a person, the interaction will drive the motors back and bring the arm to a stop.  The PR2's arms can't "punch through" an object the way traditional industrial robots can.

We have further designed the PR2's arms using relatively small motors with respect to their payload.  This is possible due to a spring counterbalance offsetting the gravity forces acting on the arms.  That is, the arms do not need to hold their own weight against gravity, so the motors need only be strong enough to hold the payload.  The arms simply can't push very hard.

In software, we have incorporated low-level checking to limit the current in a motor, the velocity of a motor, and the range through which a joint should travel.  We obviously discourage users and developers from changing these configurations.  High-level applications also avoid obstacles in navigation and movement using the various on-board sensors.  

These design choices also help make the PR2 robust.  However, a robot with the PR2’s capabilities can never be absolutely safe. Your safety as well as the safety of others critically depends on your constant attention. You must be aware of the potential dangers of using a PR2, and learn to anticipate and prevent problems.

\section{General Usage Guidelines}

While many guidelines for the safe use of a robot stem from common sense, we enumerate a basic set here.  We stress the importance of following these guidelines but re-emphasize that these guidelines alone do not guarantee safety but reduce risk.
\begin{itemize}
\item Every organization that uses a PR2 must appoint a \bf{Safety Officer}.
\begin{itemize}
\item The Safety Officer's contact information should be known by everyone in the organization who uses the PR2 (including designers, developers, programmers, and end-users).
\item Details of the Safety Officer's roles and responsibilities are described in section 2.4.
\end{itemize}
\item \bf{Before operating} or working with the PR2 you must do the following:
\begin{itemize}
\item view the safety video
\item read this User Manual, in particular the entirety of Chapter 2 on Safety
\item read and understand the latest list of potential hazards
\item know how to contact your organization's Safety Officer
\end{itemize}
\item \bf{Supervise children, visitors}, and anyone who has not followed the previous guideline.  In particular, make sure they
\begin{itemize}
\item do not come within range of the PR2 when active
\item are aware the robot could move unexpectedly and is potentially dangerous
\item are not alone with the PR2, and
\item do not operate the PR2
\end{itemize}
\item \bf{Maintain a safe environment}.  Safety is not only affected by how you operate the robot, but by the environment as well.
\begin{itemize}
\item Make sure the robot has adequate and level space for any expected or unexpected operation
\item If the robot travels on a ramp, make sure the spine is lowered and the arms are tucked in so that the center of gravity is as low as possible. The slope should not be more than 1:12. Ensure that the robot cannot fall off the edge under any circumstances.
\item Make sure the the environment is free of objects that could pose a risk if knocked, hit, or otherwise affected by the PR2
\item Make sure there are no cables or ropes that could be caught in the covers, wheels, or arms; these could pull other objects over.
\item Make sure no animals are the near the robot.
\item Keep fingers, hair, and clothing away from wheels and gears.
\item Be aware of the location of emergency exits and make sure the robot cannot block them.
\end{itemize}
\item The PR2 is designed to operate in an \bf{laboratory environment}.
\begin{itemize}
\item Do not operate the robot outdoors.
\item Do not allow the robot to come in contact with liquids (spilled drinks, rain, etc.). This is particularly important when the robot is plugged into an AC outlet .
\item Keep the robot \bf{at least 7 meters from the top of a stairway}.
\end{itemize}
\item Use extreme caution whenever the robot's spine is contracting. Objects or body parts could be crushed between the robot's body and base. Also keep people and objects away from the robot's joints.
\item Make sure the batteries cannot be overheated or punctured.  Do not run the robot without its covers. They help protect the batteries from inadvertent collisions.
\item The operator should always have immediate access to the \bf{run/stop} box and should stop the robot at the first sign of a problem. The stop button should always be pressed when examining, inspecting, or maintaining the robot.
\item Use \bf{common sense} when operating the robot.
\begin{itemize}
\item Do not allow the robot to grab or hit any person.
\item Do not allow the robot to drive into contact with, or over, any body part.
\item Do not allow the robot to interact with any sharp or dangerous items.
\item Do not allow the robot to operate potentially dangerous appliances (like stoves) or power tools.
\end{itemize}
\item Pay attention to \bf{warning labels} on the robot.
\item \bf{Do not remove the covers} of a PR2 without prior and appropriate instruction by Willow Garage.
\begin{itemize}
\item There are high voltages inside the robot.
\item There are a variety of pinch points and other mechanical dangers in the interior of the robot.
\item Counterbalances and springs store significant potential energy which could cause damage if unloaded abruptly.
\end{itemize}
\item Do not modify or remove any part of the \bf{software safety features}.
\item Regularly inspect and maintain anything the robot interacts with.  For example, regularly inspect the electrical outlets the robot plugs into.
\item \bf{Anticipate potential problems} and hazards.  Always imagine what might happen if the robot malfunctions or behaves in a way different from the desired action.  Be vigilant.
\item Be aware that many things can go wrong even with a seemingly simple activity. For example, consider what might go wrong while driving the robot with the joystick:
\begin{itemize}
\item The joystick could run out of batteries.
\item The Bluetooth connection between the joystick and the robot could fail.
\item The ROS node that talks to the Bluetooth driver could have been changed or have a bug.
\item The operating system in the robot's computer could hang.
\item The motor controller software node could fail.
\item The motor controller board could fail.
\item The PR2's electrical system could fail.
\item The motor or gear train could have a problem.
\item The wheels could have a problem.
\item Someone else could command the robot to move without your knowledge.
\item The operator could make a mistake. It's intuitive to drive a robot when it's going away from you, but when it's coming towards you everything is reversed.
\end{itemize}
\end{itemize}

\section{Safety Program}
Safety is a continual process, requiring proactive examination of the robot and its environment, and risk identification and mitigation. To facilitate this process and communication within the community, Willow Garage has implemented a safety program, spearheaded by the Safety Board.

\subsection{Willow Garage Safety Board}
The Willow Garage Safety Board identifies hazards related to Willow Garage products and ensures that appropriate actions are taken to mitigate those hazards. The Board maintains a database to keep track of the hazards and mitigation actions. To identify additional hazards for the database, the Board commissions hazard brainstorming meetings and reviews incident reports from the field. To initiate actions that reduce the severity and/or likelihood of specific hazards, the Board commissions hazard response projects. To ensure that appropriate actions remain in force, the Board commissions hazard response audits. The Board works with Safety Officers in external organizations to improve safety across all users of Willow Garage products.

The core Safety Board includes senior members of the Willow Garage management. Additional employees serve for term appointments. Members of the Board spend at least one day per month on safety related work.

\subsection{Hazard Database}
The safety database tracks information about potential hazards, which are things the robot might do that could cause damage. Example hazards include running into something or falling down stairs. Each hazard is associated with estimates of its severity and likelihood of occurrence. The database also records information on safety incidents that have been reported. Each incident is associated in the database with the hazard(s) that it exemplifies. A list of the hazards is located at safety.willowgarage.com.

\subsection{Safety Officers}

Each organization that uses Willow Garage products will appoint a Safety Officer who is responsible for all aspects of safety in the use of those products. The Safety Officer will:
\begin{itemize}
\item remain informed of all known safety hazards and mitigations,
\item ensure that all known mitigations are implemented in their organization,
\item ensure that everyone involved with the products receives safety training,
\item report any safety incidents to Willow Garage in a timely fashion, and
\item work with the Willow Garage Safety Board to improve safety.
\end{itemize}
