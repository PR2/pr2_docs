\chapter {Introduction}
This manual is intended to give you enough information to successfully install, use, and develop code on your PR2 robot.  The software on the PR2 is provided by software based on ROS.  The recommended source of information for learning about ROS and the higher-level software available for the PR2 is http://ros.org

If you wan to get started running the PR2 as quickly as possible, please start with Chapter 2 on safety (seriously - the PR2 is a big machine and can cause serious injuries or death), and then you can skip to chapter 5 to learn how to start up and run the PR2.

\section{Before you start}
\subsection{Site prep}
\paragraph{Space} To use PR2 you will need to have enough room for it to drive around and move it's arms.  The PR2 is designed to move through ADA-compliant spaces (Americans with Disabilities Act), so corridors should be at least 36" wide, doorways should be at least 32", and the ground should be flat and level.  You will need enough space for the PR2 to move around and perform tasks.
\paragraph{Safe Environment} The space where the PR2 operates should be free of hazards.  Specifically, stairways or other fall hazards can pose an extreme danger and the PR2 should not be operated near any type of dropoff.  You should also avoid hazardous objects, such as knives, sources of fire, hazardous chemicals, or furniture that could be knocked over.  See chapter 2 for more details on making sure your environment is safe.
\paragraph{Electrical} The PR2 recharges using a standard 120V American power outlet.  The robot can draw 15A of current when plugged in, so we strongly recommend recharging the PR2 only on outlets with no other devices on the circuit breaker.
\paragraph{Development tools}
You will need at least one laptop or desktop computer to use to connect to the robot.  The PR2 ships with a base-station computer, which is a desktop, but you will need to provide a screen, mouse, and keyboard.  A laptop with wireless access is ideal.
\subsection{What to know before you start}
\paragraph{Linux} We highly recommend familiarity with the Linux command-line.  The PR2 computers both run Ubuntu, and since they don't have attached displays, all tasks on them have to be performed by logging in remotely (e.g. via ssh).
\paragraph{ROS} Since all the PR2 software is based on ROS, running through the beginning tutorials available at ROS.org will help you understand the structure of the software on the robot and will give you tools to understand the software that's running and the data that is moving around in the system.
\paragraph{PR2 Safety} Be sure to familiarize yourself with the contents of the chapter on safety before using the robot.
