\chapter{PR2 Computers}
Covers the configuration of the computers and software that comes installed on the robot
\section{Computer hardware}
PR2 has two computers, each has 24 Gb of random-access memory (RAM), 2
quad-core Nehalem processors, and two attached hard-drives.  The
motherboards are XXX from Rackable systems.  For more information
about the computers themselves, please see the users's manual and
documentation at XXX (link to Rackable information).

Additionally, the PR2 ships with a basestation computer which
facilitates seamless communication with the PR2 when transitioning
between wired and wireless networks and additionally serves a role in
a number of maintenance tasks.
\subsection{Computer 1 (Master)}
Computer 1 is the computer physically located on the right side of the
robot. It is referred to as the master computer because it serves a
number of key roles for the computer infrastructure:
\begin{itemize}
\item Computer 1 stores the operating system for both computers. 
  Computer 2 cannot boot unless computer 1 has booted first.
\item Computer 1 is connected to the PR2 ethercat network, and is the
  only computer that can perform motor control.
\item Computer 1 provides routing for the rest of the robot when it is plugged
  in via the WAN port .
\item Computer 1 provides routing for the rest of the robot when connected to
  another network via an openVPN tunnel
\item Computer 1 provides DHCP services for other devices connected to the
  robot internal network.
\end{itemize}

Computer 1's PCI slot is used for a 4-port ethernet card, giving it a
total of 6 ethernet ports. These are:
\begin{itemize}
\item \texttt{lan0 - lan4}: connection to internal robot network 
\item \texttt{wan0}: connected directly to WAN port on back of robot 
\item \texttt{ecat0}: connected to robot ethercat network 
\end{itemize}

\subsection{Computer 2 (Slave)}
Computer 2 is the computer physically located on the left side of the
robot. It is referred to as the slave computer because it netboots
from computer 1.

Computer 2 only has 2 ethernet ports, \texttt{lan0} and \texttt{lan1},
both of which are connected to the internal robot network.

\subsection{Basestation}
The basestation is a zareason XXX.  It is intended to be a dedicated
point of contact through which network traffic to the robot can be
routed.  It has 2 ethernet ports, \texttt{eth0}, on the motherboard,
is the primary ethernet port.  It is intended to be plugged into your
local network.  By default it is configured to acquire an IP address
via DHCP, but can be given a static IP address if desired.
\texttt{eth1}, on the pci card is a dedicated port for servicing the
robot.  When necessary, this port should be plugged directly into the
robot service port.

When configured properly, it is important that the basestation is
visible on port \texttt{1194} both via your local wired network as
well as via your local wireless network.  This will likely require
assistance from your local network administrator.

\section{Networking}
The majority of communication between components onboard the PR2
happens via an onboard ethernet network.  This onboard network can be
accessed directly via either wired or wireless connections.
Additionally, the robot can be accessed via the basestation through a
VPN tunnel.
\subsection{Network segments}
\begin{itemize}
\item \texttt{10.68.0.0}: Robot Internal Network. The
  \texttt{10.68.0.0} subnet is the primary network used internally by
  the robot.  Both computers have addresses on this subnet.  All
  ethernet-based devices, such as the cameras and the power board, are
  given addresses on this subnet.  The robot Service Port, and the
  cradlepoint ctr350 Wireless Access Point are directly connected to this
  network.  The master computer, if booted, will give out IP addresses
  via DHCP in the range \texttt{10.68.0.100-10.68.0.199}.  Important
  addresses on this network:
  \begin{itemize}
  \item \texttt{10.68.0.1} - c1 (master computer)
  \item \texttt{10.68.0.2} - c2 (slave computer)
  \item \texttt{10.68.0.5} - wifi-router (linksys router)
  \item \texttt{10.68.0.6} - reserved for basestation on local network
  \item \texttt{10.68.0.91} - c1-esms (master computer enterprise server management system)
  \item \texttt{10.68.0.92} - c2-esms (master computer enterprise server management system)
  \item \texttt{10.68.0.250} - wap (wireless access point)
  \end{itemize}
\item \texttt{10.68.X.0}: Robot VPN Network. The primarily role of the
  basestation is to function as a VPN server for the robot.  Each
  robot can be given a unique VPN subnet to facilitate the operation
  of multiple robots using a single VPN server.  Under normal
  operation the basestation will automatically forward relevant
  traffic into the VPN network, hiding this from the end-user.
  However, if greater security is desired, the basestation can instead
  be configured to require users to be assigned a key to access the
  robot VPN network.  Important addresses on this network:
  \begin{itemize}
  \item \texttt{10.68.X.1} - c1 (master computer)
  \item \texttt{10.68.X.2} - c2 (slave computer)
  \item \texttt{10.68.255.1} - basestation
  \end{itemize}
\end{itemize}

\subsection{Service Port} The robot service port is the bottom
ethernet port on the back panel of the robot.  It connects directly to
the robot internel network, allowing you to connect to the computers
directly.  \textit{DO NOT PLUG THIS PORT INTO YOUR LOCAL NETWORK.}
The master computer serves DHCP for this network and if it conflicts
with another DHCP server this will most likely cause problems both on
your local network and on the robot network depending on which DHCP
server takes precedence.
\subsection{Wireless access point} The robot comes configured with a
cradlepoint ctr350 configured as a wireless access point
\href{http://www.cradlepoint.com/products/ctr350-mobile-broadband-router}.
The ESSID of this network defaults to PRLAN, and allows direct access
to the Robot Internal Network.
\subsection{WAN port} The robot WAN port is the top ethernet port on
the back panel of the robot.  This connects directly to \texttt{wan0}
on the master computer.  This port is intended to be plugged into your
local network.  The robot will attempt to acquire an IP address via
DHCP, and then attempt to contact the basestation at a known IP
address.
\subsection{Wireless router}
The PR2 has a
\href{http://www.linksysbycisco.com/US/en/products/WRT610N}{Linksys
  WRT610N} dual-N band wireless router.  This router can be configured
to connect to your local wireless network.  In the absence of a WAN
connection, the robot will attempt to contact the basestation through
the wireless router instead.

\section{OS}
The operating system running on the PR2 computers is an extended
version of Ubuntu 9.04 (Jaunty Jackalope). It depends on a number of
additional packages for system configuration, but should otherwise be
familiar to for Ubuntu users. If you run into a computer problem not
covered by the PR2 documentation, the
\href{https://help.ubuntu.com/9.04/index.html}{Ubuntu Documentation}
is the next place to look.

\subsection{NFS and Unionfs}
The single largest difference between a normal Ubuntu installation and
the PR2 configuration is that the Slave computer mounts nearly its
entire filesystem via NFS.  The exceptions to this are the directories
\texttt{/etc}, \texttt{/var}, and \texttt{/pr2bin} which are mounted
via unionfs-fuse.  In short, unionfs allows one to specify an
additional overlay on top of the underlying filesystem.  The contents
of this overlay can be found in the directory \texttt{/slave} on the
master machine.  Files added here will show up in the appropriate
location on the slave machine.  For more information on how this is
set up, see the man page for \texttt{unionfs-fuse} and look at the
init-script \texttt{unionfs-fuse-nfs-root}.

New software and configuration changes should only be made on the
master machine.  Since the slave machine mounts most of its
filesystems read-only, this will usually be enforced for you.  If you
are using a non-standard piece of software that attempts to write to
the filesystem outside of your home directory you will likely need to
make accommodating changes to either the computer configuration or the
software you are trying to run.

\subsection{autofs}
Both of the computers have automount configured for mounting the
\texttt{/home} partition of the other computer in a
computer-independent way using autofs. These automounts are located in
the directory \texttt{/pr}. To get to the home partition on computer 1
you can use the path \texttt{/pr/1/} and to get to the home partition
on computer 2 you can use the path \texttt{/pr/2/}. You should rarely
need to do this explicitly, but it is necessary to make sense of home
directory locations.

\subsection{Home directories}
The default configuration for user home directories (as given in
\texttt{/etc/passwd} is \texttt{/u/username}.  Instead of a directory,
this location is by default a symlink to \texttt{/pr/1/username}.  To
place a users home directory at a different location, such as the disk
on the slave computer, an admin can simply move (or copy) the home
directory and update the symlink accordingly.

\subsection{Kernel}
Discuss RT\_PREEMT and anything else non-vanilla about the kernel

\subsection{Storage}
Each computer has two hard-drives - one internal 2.5" 500Gb \href{http://www.seagate.com/www/en-us/products/laptops/momentus/momentus_7200.4_g_force/}{Seagate Momentus} drive, and a
slot for a removable 3.5" SATA hard-drive that is exposed on the top
of the robot base.

By default the only hard-drive used is the internal drive on the
Master computer.  There are 3 relevant partitions created during
installation:
\begin{itemize}
\item  \texttt{c1:/dev/sda1} -- \texttt{/} -- holds the root filesystem for the OS
\item  \texttt{c1:/dev/sda5} -- \texttt{c1:/home} -- stores user home directories (linked to by \texttt{/u} by way of \texttt{/pr/1})
\item  \texttt{c1:/dev/sda6} -- \texttt{/hwlog} -- stores hardware logs generated by the pr2
\end{itemize}

The internal hard-drive on the Slave computer has a single partition,
which is generally used as extra user storage:
\begin{itemize}
\item  \texttt{c2:/dev/sda1} -- \texttt{c2:/home} -- stores user home directories (linked to by \texttt{/u} by way of \texttt{/pr/2})
\end{itemize}

Finally, both computers are configured to make use of the additional
removable drives.  Any drive loaded into the removable bay will always
show up as \texttt{/dev/removable}.
\begin{itemize}
\item \texttt{/dev/removable1} -- \texttt{/removable} -- stores
  temporary data users may want to move off the robot, primarily used
  for large bag files.  This NOT mounted by default.  Users must
  explicitly
\begin{verbatim}
$ mount /dev/removable
\end{verbatim}
to use it, and should 
\begin{verbatim}
$ umount /dev/removable
\end{verbatim}
when done.
\end{itemize}

\subsection{User accounts}
\label{creating accounts}
The pr2 ships with several default user accounts, which are used for
system utilities, maintenance, running demos, etc.  These accounts
are:
\begin{itemize}
\item{pr2\_admin}
\item{pr2\_demo}
\end{itemize}
In addition to these accounts, the expectation is that each person
logging into the robot to develop code will have their own user
account.  Any robot admin can create a new user using the
\texttt{\href{http://unixhelp.ed.ac.uk/CGI/man-cgi?adduser}{adduser}} command (not the \texttt{useradd} command).

Some examples:
\begin{verbatim}
$ sudo adduser bob
$ sudo adduser bill --shell /usr/bin/tcsh --uid 2000
\end{verbatim}

\textit{Note: it may be helpful to assign users the same UID used on
  your local network so the UIDs are consistent when mounting shares
  or moving around the removable drives.}

Moving the home directory and creating the symlink is handled
automatically by the script: \texttt{/usr/local/sbin/adduser.local}.

\subsection{User groups}
There are a couple of important groups on the pr2:
\begin{itemize}
\item \texttt{admin} -- Members of this group have full root privileges when using the sudo command
\item \texttt{pr2-admin} -- Members of this group have access to change pr2-specific configuration settings
\item \texttt{apt} -- Members of this group can install new software
\end{itemize}

To add a user to a group, use the \texttt{\href{http://unixhelp.ed.ac.uk/CGI/man-cgi?usermod}{usermod}} command.  The most
common invocation is \texttt{-a} (append) \texttt{-G} (group).  For example:

\begin{verbatim}
$ sudo usermod -a -G admin bob
\end{verbatim}

\subsection{Backing up and restoring users}
Before reinstalling the robot operating system you will likely want to
back up the user accounts. This can be done with the command:
\texttt{pr2-usermigrate}. To save users:
\begin{verbatim}
$ sudo pr2-usermigrate save myrobot.users
\end{verbatim}

Move this file off the robot before reinstalling.  Then, to restore users:
\begin{verbatim}
$ sudo pr2-usermigrate load myrobot.users
\end{verbatim}


\subsection{Clock synchronization}
Consistent time-stamping of data from the two computers is important
for interpreting sensor data on a moving system.  As a result, keeping
the system time on the two computers synced together requires some
attention.  The system that is used for this is
\href{http://chrony.tuxfamily.org/}{chrony}, and the general strategy
is to have the two computers tightly coupled to one another, but
loosely coupled to an external time source to prevent the robot time
from drifting too far from the outside world.

\section{System installs}
When the robots ship, the most recent version of ROS and the PR2
software stacks is installed in /opr/ros.  By default, new users will
have a ROS installation which references the pre-installed code.
Users who wish to modify or replace a part of the system are
recommended to install a development version of the stacks or packages
which they wish to modify, but to continue to use the base
installation for most of the robot functionality.

The pr2\_admin user can install updated versions of stacks into this
globally visible space.


