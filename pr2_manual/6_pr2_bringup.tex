\chapter{Running PR2}
These are the instructions for running pr2.  Depending on organization there are two options for this chapter.  One is to have this be basically a narrated index into the tutorials that fill the rest of the manual.  The other is to have this section walk someone through turning the robot on and using it to do a variety of things in a linear manner.
\section{out of the box}
Describes how to get the pr2 out of the box, including all the first-time checkout and startup stuff.  Doesn't involve putting a new account on the robot.  Basically, TODO list to get it charging in the corner. Target - someone who has never touched a robot before, on a pristine robot.
\subsection{Opening the crate}
\subsection{What's in the box}
\subsection{Charging the batteries}
First, identify an appropriate socket to use for charging the robot.  Make sure the electrical socket has enough capacity (don't put it on the same circuit with other big devices like computers).   See ref{powersystem} for more information on what the requirements are for the power socket.

\section{Turning pr2 on}
Turn on the PR2 by switching the red rocker-switch DC Breaker next to the plug to the on position.  You should hear the power board spin the fan up, then some beeping as the computers start up.  See ref{computers} for more information on the boot-up sequence and how to interpret the beeps and LEDs on the front cover.

\subsection{Configuring the network}
If the PR2 hasn't been set up in your environment before, you'll need to set up the base station and configure the network on the robot to work with your wireless environment ref{configuring network}.  The instructions from this point on assume that your network is configured and that you're working from a computer that has network connectivity to the robot.

\subsection{Logging in}
To log into the robot, you'll need an account.  Ask your site administrator to make an account for you ref{creating accounts}.  Once you have an account and a password on the robot, just log in with
\begin{verbatim}
ssh username@prx1
\end{verbatim}

\subsection{Starting the robot}
To start up the whole robot, you will want to run the launch file in /etc/pr2/pr2.launch.  You can configure your environment to run this yourself, but for now just ask the system to start it up for you:
\begin{verbatim}
pr2 start
\end{verbatim}

subsection{Debugging and monitoring your progress}
On the machine you're connecting from, you will also want to have a ROS tree built so that you can use the debugging and visualization tools.  The base-station comes with this pre-installed, or follow the installation instructions at www.ros.org to install ROS on your own computer.
Once you've installed ROS, build the package pr2\_dashboard:
\begin{verbatim}
rosmake pr2_dashboard
\end{verbatim}

Once pr2\_dashboard is built, set your ROS\_MASTER\_URI to point to the master on the robot and start the dashboard:
\begin{verbatim}
export ROS\_MASTER\_URI=http://prx1:11311
rosrun pr2\_dashboard pr2\_dashboard
\end{verbatim}



Get a laptop with wireless access (or use the base-station computer)

Connect to PRXLAN wireless network, served by the robot (see section on network configuration for more information)

SSH in to the first computer with the pr2\_demo account
\begin{verbatim}
ssh pr2_demo@prx
\end{verbatim}


\subsection{Diagnostics}
\section{driving pr2 around}
\section{visualizing sensor data}
Explain how to run rviz
\section{moving the arms}
Describe how to run tuck\_arms.

Describe how to send the arm to a specific set of joint angles.
\section{building a map}
Does this even belong here?
\section{autonomous navigation}
Does this even belong here?
