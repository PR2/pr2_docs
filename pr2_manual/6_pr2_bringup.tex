\chapter{Running PR2}
These are the instructions for running pr2.  Depending on organization there are two options for this chapter.  One is to have this be basically a narrated index into the tutorials that fill the rest of the manual.  The other is to have this section walk someone through turning the robot on and using it to do a variety of things in a linear manner.
\section{out of the box}
Describes how to get the pr2 out of the box, including all the first-time checkout and startup stuff.  Doesn't involve putting a new account on the robot.  Basically, TODO list to get it charging in the corner. Target - someone who has never touched a robot before, on a pristine robot.
\subsection{opening the crate}
\subsection{what's in the box}
\subsection{charging the batteries}
First, identify an appropriate socket to use for charging the robot.  Make sure the electrical socket has enough capacity (don't put it on a circuit with other big devices like computers).   See ref{powersystem} for more information on what the requirements are for the power socket.

\section{turning pr2 on}
\subsection{plugs and power switches}
Turn on the PR2 by switching the red rocker-switch next to the plug to the on position.  You should hear the power board spin the fan up, then XXX, then XXX.  The computers are booted when the LEDs on the drive-bays have the XXX light on.


\subsection{Connecting over the web}
By default when the robot boots, the web server should already be running

\subsection{Logging in}
Get a laptop with wireless access (or use the base-station computer)

Connect to PRXLAN wireless network, served by the robot (see section on network configuration for more information)

SSH in to the first computer with the pr2\_demo account
\begin{verbatim}
ssh pr2_demo@prx
\end{verbatim}


\subsection{Diagnostics}
\section{driving pr2 around}
\section{visualizing sensor data}
Explain how to run rviz
\section{moving the arms}
Describe how to run tuck\_arms.

Describe how to send the arm to a specific set of joint angles.
\section{building a map}
Does this even belong here?
\section{autonomous navigation}
Does this even belong here?
